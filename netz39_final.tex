%%
%% This is file `beispiel.tex',
%% generated with the docstrip utility.
%%
%% The original source files were:
%%
%% g-brief.dtx  (with options: `beispiel')
%% 
%% =======================================================================
%% 
%% Copyright (C) 1991-2003 Michael Lenzen.
%% 
%% For additional copyright information see further down in this file.
%% 
%% This file is part of the G-BRIEF package.
%% -----------------------------------------------------------------------
%% 
%% It may be distributed under the terms of the LaTeX Project Public
%% License LPPL), as described in lppl.txt in the base LaTeX distribution.
%% Either version 1.1 or, at your option, any later version.
%% 
%% The latest version of this license is in
%% 
%%          http://www.latex-project.org/lppl.txt
%% 
%% LPPL Version 1.1 or later is part of all distributions of LaTeX
%% version 1999/06/01 or later.
%% 
%% 
%% Error reports in case of UNCHANGED versions to
%% 
%%                            <lenzen@lenzen.com>
%%                            <m.lenzen@t-online.de>
%% 
%% 
\def\filedate{2008/07/15}
\def\fileversion{4.0.2}

\documentclass[12pt,ngerman,a4paper,utf8,sans,t1]{g-brief-ntz39}

\usepackage[ngerman]{babel}

\lochermarke
\faltmarken
\fenstermarken
\unserzeichen
 \trennlinien
%% \klassisch

\begin{document}

\Unterschrift        {Manni Mitglied}

\Postvermerk         {}      % E I N S C H R E I B E N}
\Adresse             {Frau und Herr\\
                      Willi Geldgeber\\
                      Fass Ohne Boden 1\\
                      \\
                      D-12345 Goldstadt
                      }

\Datum               {\today}
\IhrZeichen          {}
\IhrSchreiben        {}
\MeinZeichen         {Netz39}

\Betreff             {Foo f\"ur die Foobar}

\Anrede              {Sehr geehrte willige Geldgeber,}
\Gruss               {Mit freundlichen Gr\"u\ss{}en}{1cm}

\Anlagen             {}
\Verteiler           {}

\begin{g-brief}

abcdefghijklmnopqrstuvwxyz

ABCDEFGHIJKLMNOPQRSTUVWXYZ

äöüß

1234567890


\end{g-brief}
\end{document}


\endinput
%%
%% End of file `beispiel.tex'.
